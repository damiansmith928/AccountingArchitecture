\documentclass[12pt]{article}
\usepackage[left=1in, right=1in, top=1in, bottom=1in]{geometry}
\usepackage{color}
\usepackage{setspace}
\usepackage{graphicx}
\tolerance=1
\emergencystretch=\maxdimen
\hyphenpenalty=10000
\hbadness=10000
\frenchspacing{}
\clubpenalty 10000
\widowpenalty 10000
\setcounter{secnumdepth}{0}
\setlength{\parindent}{1cm}

\begin{document}
\section \today
\section {From: Damian Smith}
\section {To: Professor Coyne}
\section{ACCT 4020: Explanation of revisions to the "AA Framework" article}

\indent I have proposed two additions to the article, although they are related to one another.  
The Accounting Architecture article, and throughout the Accounting Systems course, there has been a heavy emphasis placed upon UNIX, 
and Linux based operating systems without giving much explanation as to how accountants can still utilize the systems that they may be most aquatinted with.  

Firstly, within the 'Software' section on pages 22-23 I would add that OS X and Microsoft can still offer benefits to accountants who may need to act both as accounting professional, and be well versed
 in IT within their firm.  Many organizations and smaller firms may not have the resources to staff a large IT department, and therefore the accounting team 
 will find themselves needing to quickly adapt to changes and technological needs in order to perform their roles as accountants, to provide information. I believe that it is a mistake to limit the frontier that is Accounting Architecture to one system, while there may be many systems that may provide solutions.

I have also made an addition to the proposed curriculum changes, within the lower-level course work section on page 32.  With IT and computer science being a sometimes intimidating topic for those who may not already be familiar with the terminology, and implementation of 
such enterprise-level technologies, there is a need for courses that directly relate to how accountants will work with technology.  Currently there is a lack of 
actual hands-on learning available, and much of the course work within the undergraduate accounting major feels rather segmented already.  Before implementing a new AA curriculum path, I would suggest 
evaluating the balance of "hands-on" learning opportunities with "text book" style learning.  Just like with any other new language one sets out to learn, full immersion is essential to learning the lingo and language of technology, software deployment and system design.
\end{document}